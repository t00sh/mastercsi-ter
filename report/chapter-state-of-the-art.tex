\chapter{État de l'art des attaques sur SSL/TLS}

\begin{figure}[H]
  \caption{Historique des attaques SSL/TLS (source : Cloudflare \cite{cloudflare})}
  \fbox{\includegraphics[width=\textwidth]{./history-tls-attacks.png}}
\end{figure}

\section{Attaques liées à l'implémentation}

\subsection{Heartbleed (2014)}
Cette faille est une faille d'OpenSSL basée sur SSL/TLS. Cette vulnérabilité permet de lire la mémoire du serveur. Cette faille est due à une erreur de programmation dans la focntion Heartbeat qui permet de maintenir la connexion sécurisée entre le navigateur et le serveur.

En effet, l'attaquant peut mentir sur le poids de la requête qu'il envoit et obtenir des données qui ne lui était pas destinées mais qui se trouve en mémoire. Il peut donc récupérer les informations comme des identifiants, des mots de passes, des cookies de sessions ou encore des clés de chiffrements \cite{heartbleed}.

\subsection{BERserk (2014)}

Blablabla \cite{berserk}

\section{Attaques liées à la cryptographie}

\subsection{BEAST (2011)}
L'attaque BEAST, Browser Exploit Against SSL/TLS concerne SSL 3.0 et TLS 1.0. L'attaquant, placé en Man-in-the-Middle, peut déchiffrer les données échangées entre le serveur et le client grâce à une vulnérabilité d'implémentation du mode CBC, Cipher Block Chaining. L'attaque se fait côté client en injectant des paquets dans le flux TLS. Cette technique est qualifiée d'attaque à clair connu. La faille du protocole réside dans le chiffrement par bloc. En effet, les blocs sont chiffrés les uns après les autres en utilisant le précédent.

Voici le déroulement de l'attaque :

\begin{enumerate}
\item Injection du code chez la victime
\item La victime envoie de données forgées via SSL
\item L'attaquant écoute le trafic
\item Il renvoit les informations nécessaires à son code injecté
\item La victime réenvoit les données forgées et on repète les deux étapes précédentes
  \item L'attaquant dérive tous les cookies
\end{enumerate}
\cite{beast}

\subsection{Cassage de RC4 (2013)}

Blablabla \cite{rc4}

\subsection{Lucky13 (2013)}

Blablabla \cite{lucky13}

\subsection{Logjam (2015)}

Blablabla \cite{logjam}

\subsection{DROWN (2016)}

Blablabla \cite{drown}

\subsection{SLOTH (2016)}

Blablabla \cite{sloth}

\subsection{SWEET32 (2016)}

Blablabla \cite{sweet32}

\subsection{ROBOT (2017)}

Blablabla \cite{robot}

\section{Attaques sur le protocole}

\subsection{CRIME (2012)}

\cite{crime}
CRIME (compression ratio info-leak made easy), utilise une faille de l'algorithme de compression utilisé par TLS. Cette attaque vise surtout les cookies de sessions. L'idée est d'envoyer des caractères aléatoires dans un cookie forgé et de comparer la compression de ce dernier avec le cookie originel du client. Si le cookie forgé est partiellement compressé, on peut inférer qu'une partie de ce dernier correspondant au cookie du client. On procède ainsi par étapes successives pour retrouver le cookie et voler la session de la victime.

Cette attaque nécessite que l'attaquant se place en MITM pour pouvoir observer la taille du chiffré envoyé par le client mais également pouvoir envoyer des requêtes forgées par l'attaquant lui-même.

\subsection{BREACH (2012)}

Blablabla \cite{breach}

\subsection{POODLE (2014)}

\cite{poodle}

L'anagramme POODLE signifie "padding oracle on downgraded legacy encryption". L'attaque concerne surtout les chiffrements SSL 3.0, bien qu'une variante de la faille fut trouver sur TLS un peu après.
Le padding dont parle le nom de l'attaque vient des méthodes de chiffrements cryptographiques. Lorsqu'un message envoyé un trop court pour l'algorithme de chiffrement, on lui rajoute une partie arbitraire, le "padding" pour qu'on puisse appliquer l'algorithme dessus.

Bien que le TLS soit un protocol plus sûr que SSL et recommandé aujourd'hui, de nombreux serveurs continuent d'accepter le SSL 3.0 si une connexion par TLS venait à être impossible. L'attaque POODLE se sert de ce fait pour forcer la connexion à se faire en SSL.

La vulnérabilité en elle-même vient de la façon dont sont encodés les blocs de donnés avec SSL. L'attaquant a besoin de deux choses pour se faire :
\begin{enumerate}
    \item L'attaquant doit pouvoir changer une partie du message injecté par le client dans l'algorithme
    \item Il doit avoir un retour sur le texte chiffré par SSL.
\end{enumerate}

Ces deux conditions peuvent s'exécuter avec par exemple une attaque en homme du milieu. L'attaquant doit toutefois avoir un contrôle également sur le client pour modifier le padding, ce qui est une tâche supplémentaire à effectuer.

En modifiant le padding, l'attaque permet de récupérer un byte d'information du message initial en envoyant un maximum de 256 requêtes. Cela peut ammener l'attaquant à envoyer énormément de réquêtes pour obtenir le message dans son intégralité.

L'avantage de cette attaque est qu'elle ne nécessite pas de connaître la clef utilisée pour le chiffrement.

\subsection{FREAK (2015)}

Blablabla \cite{freak}

\subsection{WeakDH (2015)}

Blablabla \cite{weakdh}

\section{Attaques de l'homme du milieu}

\subsection{SSLsniff (2002)}

Blablabla \cite{sslsniff-website}

\subsection{SSLstrip (2009)}

Blablabla \cite{sslstrip-website}

\subsection{HTTPS interception}

Blablabla \cite{https-interception}
