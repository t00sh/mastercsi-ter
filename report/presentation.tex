\documentclass{beamer}
\usepackage{graphicx}
\usepackage{array}
\usepackage{amsmath, nccmath}
\usepackage[utf8]{inputenc}
\usepackage[T1]{fontenc}
\usepackage[french]{babel}

\usetheme{Boadilla}

%%% Metadata %%%
\title[Attaques sur HTTPS]{Attaque Man-In-The-Middle de HTTPS}
\subtitle{SSLStrip, HTTPS-Interception...}
\author[S. Duret - A. Risi - B. Guevel]{Simon Duret\\Amélie Risi\\Brendan Guevel}
\institute[]{Université de Bordeaux}
\date{14 Mai 2018}

\begin{document}

%%% Title page %%%
\begin{frame}
  \titlepage
\end{frame}

\begin{frame}{Introduction}
    {\Large \centerline{HTTP est \textbf{le} protocole central du net}}

    Assure la transmission des données entre un client et un serveur \\
    % Dire à l'oral que client = navigateur généralement, et serveur = site web
    -> Problème : Tout le traffic est en clair \\
    % Donner des exemples à l'oral : interception du wifi, ou physiquement au niveau du cable ethernet, ...
    HTTPS : Extension de HTTP pour chiffrer un échange entre un client et un serveur \\
    -> Assure la confidentialité des données (mot de passe, numéro de carte bancaire, ...) \\

    Malgré ce chiffrement, des attaques restent possibles...

\end{frame}

\begin{frame}{Introduction}
    {\Large \centerline{Plan}}

    1. Fonctionnement de HTTP et HTTPS \\
    2. Etat de l'art des attaques sur SSL/TLS \\
    3. Presentation de notre implémentation \\
        a. Environnement qemunet \\
        b. SSLstrip \\
        c. SSLstrip+ \\
        d. SSLstrip NTP \\
        e. HTTPS interception \\

\end{frame}

\section{Fonctionnement de HTTP}

\begin{frame}{Fonctionnement de HTTP}
    Procotole de communication client-serveur \\
    Le client demande une ressource à un serveur via une requête HTTP \\
    % Généralement une page html
    Le serveur envoie au client la ressource demandée en réponse \\
    Un navigateur web permet d'automatiser ce processus \\
    De plus ce dernier permet d'afficher la page avec la mise en forme html/css

\end{frame}

\begin{frame}{Exemple d'une requête HTTP}
    \includegraphics[scale=0.50]{../medias/perdu.png}
    % Faire une petite démo en live (avec la même requête)
\end{frame}

\begin{frame}{HTTPS : la version chiffrée de HTTP}


\end{frame}

\section{Etat de l'art}

\begin{frame}{Historique des attaques sur SSL/TLS}
    \includegraphics[scale=0.19]{../medias/history-tls-attacks.png}
\end{frame}

\section{Environnement}

\begin{frame}{Environnement}
    \begin{block}{Bloc simple}
        Toto
    \end{block}
\end{frame}

\section{Attaque SSLstrip}
\begin{frame}{Attaque SSLstrip}
    \includegraphics[scale=0.4]{../medias/sslstrip/attack.png}
\end{frame}

\section{Attaque SSLstrip+}
\begin{frame}{Attaque SSLstrip+}
    \includegraphics[scale=0.32]{../medias/sslstrip2/attack.png}
\end{frame}

\section{Attaque SSLstrip NTP}
\begin{frame}{Attaque SSLstrip NTP}
    \includegraphics[scale=0.32]{../medias/sslstrip-ntp/attack.png}
\end{frame}

\section{Attaque HTTPS interception}
\begin{frame}{Attaque HTTPS interception}
    \includegraphics[scale=0.4]{../medias/https-interception/attack.png}
\end{frame}







\end{document}
