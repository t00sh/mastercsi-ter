%%%%%%%%%%%%%%%%%%%%%%%%%%%
% Introduction            %
%%%%%%%%%%%%%%%%%%%%%%%%%%%
\begin{frame}{Introduction}

  {\Large \centerline{HTTP est \emph{le} protocole central du net}}
  \hspace{40cm}

  \begin{columns}
    \begin{column}{0.5\textwidth}
      \begin{itemize}
        % Dire à l'oral que client = navigateur généralement, et serveur = site web
      \item{Assure la transmission des données}
      \item{Architecture client/serveur}
      \end{itemize}
      \begin{alertblock}{Problème}
        \begin{itemize}
        % Donner des exemples à l'oral : interception du wifi, ou physiquement au niveau du cable ethernet, ...
        \item{Tout le trafic est en clair}
        \end{itemize}
      \end{alertblock}

      \begin{exampleblock}{Solution}
        \begin{itemize}
        \item{HTTPS : extension de HTTP}
          % (mot de passe, numéro de carte bancaire, ...)
        \item{Confidentialité et authentification}
        \end{itemize}
      \end{exampleblock}
    \end{column}
    \begin{column}{0.5\textwidth}
      \includegraphics[width=\linewidth]{../medias/www.png}
    \end{column}
  \end{columns}

  \hspace{20cm}

  {\Large \centerline{Malgré cette protection, des attaques restent possibles...}}


\end{frame}
